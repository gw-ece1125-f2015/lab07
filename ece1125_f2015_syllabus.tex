\documentclass{article}
\title{ECE 1125 -- Data Structures and Algorithms \\ Fall 2015 \\ The George Washington University}
\author{Lecturer: Ahsen J. Uppal (\texttt{auppal@gwu.edu})
\\ Teaching Assistants: Engin Kayrakliogu (\texttt{engin@email.gwu.edu}), \\
Yang Hu (\texttt{huyang@email.gwu.edu}), Yongbo Li (\texttt{lib@email.gwu.edu})}

\date{}
\begin{document}

\maketitle

\section{Introduction}
In order to design and develop useful software applications, we have
to rely on fundamental building blocks for organizing, storing,
managing, and manipulating information. This course provides a
hands-on introduction to using data structures to effectively store,
and manage information and algorithms to efficiently manipulate that
information. As we will see, these concepts are closely related.
\\

The prerequisite is ECE 1120 Introduction to C Programming.
\\

\noindent
\textbf{Students are required to honor the GWU Code of Academic Integrity when completing all assignments, projects, and examinations.}

\subsection{Textbook}
Horowitz, Sahni, Anderson-Freed, \textit{Fundamentals of Data Structures} (2nd Edition).

\subsection{Contact Information}
The instructor may be contacted at \texttt{auppal@gwu.edu} and the
teaching assistant at \texttt{engin@email.gwu.edu}. The instructor's
office hours are Tuesdays and Thursdays from 4:00pm to 5:00pm, and
other times by appointment.


\section{Grading}
\begin{center}
\begin{tabular}{|l|l|}
\hline
Participation & 10\% \\
Homeworks and Lab Assignments & 25\% \\
Project 1 &     15\% \\
Midterm   &     15\% \\
Project 2 &     15\% \\
Final Exam &    20\% \\
\hline
\end{tabular}
\end{center}

\subsection{Policies}
\begin{itemize}
\item Participation: Each student must be willing to participate fully in the class. It is insufficient to just show up to class. 
\item Assignments: You will be assigned small homework and lab assignments to complete under guidance from the teaching assistant.
\item Projects: There will be two major projects during the semester, each requiring a significant amount of time to finish. You are strongly encouraged to talk to the instructors often and seek help as soon as possible.
\item \textbf{Late Penalty: Note that there is a 20\% per day penalty for each day an assignment is late.}  
\end{itemize}


\newpage
\section{Tentative Schedule}
The course will tentatively follow the schedule outlined in the textbook:

\begin{table*}[htdp]
\begin{center}
\begin{tabular}{|c|l|c|}
\hline
{\em Week} & {\em Topic} & {\em Notes} \\
\hline
\hline
1  & Introduction to Algorithms and Data Organization & \\
   & (analysis, space and time complexity) & \\
2  & Arrays and Strings & \\
3  & Stacks and Queues & \\
4  & Linked Lists (singly and doubly linked) & Project 1 Assigned \\
5  & Trees (basic facts, binary trees) & \\
6  & Trees (search, heap) & \\
7  & Graphs (basic facts, representations) & Midterm Exam 10/15 \\
8  & Graphs (shortest paths, spanning trees, topological sorting) & Project 1 Due \\
9  & Graphs & \\
10 & Internal Sorting (insertion, quick, and merge) & \\
11 & Internal Sorting (heap, radix) & \\
12 & Hashing & \\
13 & Advanced Topics & No class 11/26 \\
14 & Advanced Topics & \\
15 & Advanced Topics & \\  
16 & Final Exam & Project 2 Due \\

\hline
\end{tabular}
\end{center}
\end{table*}%

\end{document}
